% This file defines the various symbols customization for the book

% Packages
\usepackage{standalone, amsmath, amssymb, graphicx, color}
\usepackage{url}
\urlstyle{same}
\usepackage{caption}
\captionsetup{font=scriptsize,labelfont=scriptsize}
\usepackage[scriptsize,tight]{subfigure}

% Math
% ====
\newcommand{\vo}[1]{\boldsymbol{#1}} % Vector object
\newcommand{\mo}[1]{\boldsymbol{#1}} % Matrix object

%\newcommand{\vo}[1]{{#1}} % Vector object
%\newcommand{\mo}[1]{{#1}} % Matrix object

\newcommand{\real}{\mathbb{R}}
\newcommand{\integer}{\mathbb{Z}}
\renewcommand{\natural}{\mathbb{N}}
\newcommand{\complex}{\mathbb{C}}
\newcommand{\pderiv}[2]{\frac{\partial #1}{\partial #2}}
\newcommand{\oderiv}[2]{\frac{d #1}{d #2}}

% For defining matrices use environment  amsmath \bmatrix
% For defining vectors use environment  amsmath \pmatrix

% \newtheorem{theorem}{Theorem}
% \newtheorem{corollary}[theorem]{Corollary}
% \newtheorem{lemma}[theorem]{Lemma}
% \newtheorem{proposition}[theorem]{Proposition}
% \newtheorem{definition}[theorem]{Definition}
% \newtheorem{remark}[theorem]{Remark}
% \newtheorem{assumption}{Assumption}

\newcommand{\Trace}{\mathbf{Tr}}

% Statistics
% =======
\newcommand{\Exp}[1]{\boldsymbol{\mathsf{E}} \left[#1\right]}
\newcommand{\Var}[1]{{\bf Var}\left[#1\right]}
\newcommand{\mean}[1]{\vo{\mu}_{#1}}
\newcommand{\Sig}[2]{\vo{\Sigma}_{#1#2}}
\newcommand{\Om}{\Omega} % Domain of pdf
\newcommand{\PDF}[1]{p_{#1}(#1)} 
\newcommand{\pdf}{p(\cdot)} 
\newcommand{\pdfx}{p(t,\x)} 
\newcommand{\pdfy}{p(t,\y)} 
\newcommand{\pdfp}{p(\param)} 
\newcommand{\prob}[1]{\text{\bf P}\left(#1\right)} 
\newcommand{\uniform}[1]{\mathcal{U}_{#1}}
\newcommand{\gaussian}[2]{\mathcal{N}(#1,#2)}
\newcommand{\normal}[2]{\mathcal{N}(#1,#2)}
\newcommand{\FP}{\mathcal{P}\;}
\newcommand{\FPt}{\mathcal{P}_t\hspace{0.5mm}}

% Dynamics
% ========
\newcommand{\x}{\vo{x}} % state
\newcommand{\xdot}{\dot{\vo{x}}} % state derivative
\newcommand{\y}{\vo{y}} % output
\renewcommand{\u}{\vo{u}} % output
\newcommand{\param}{\vo{\Delta}} %parameters
\newcommand{\domain}[1]{{\mathcal{D}_{#1}}} %parameters
\newcommand{\Y}{\vo{Y}} % Multiple observations
\newcommand{\w}{\vo{w}} % Process noise
\newcommand{\n}{\vo{n}} % Sensor noise
\newcommand{\pnc}{\vo{Q}} % pnc = process noise covariance
\newcommand{\snc}{\vo{R}} % snc = sensor noise covariance
\newcommand{\A}{\mo{A}} % A matrix of linear system
%\renewcommand{\B}{\mo{B}} % B matrix of linear system
\let\B=\undefined
\newcommand{\B}{\mo{B}} % B matrix of linear system

%\newcommand{\basis}[2]{\phi_{#1}({#2})}% Basis functions
\usepackage{xifthen}% provides \isempty test
\newcommand{\basis}[2]{%
 \phi_{#1}
  \ifthenelse{\isempty{#2}}%
    {}% if #1 is empty
    {({#2})}% if #1 is not empty
}

%\newcommand{\diag}{\bf diag} % B matrix of linear system
\newcommand{\diag}{\boldsymbol{\mathsf{diag}}}


\newcommand{\xpc}{\x_{pc}} % state
\newcommand{\xpcdot}{\dot{\x}_{pc}} % state
\newcommand{\ypc}{\vo{y}_{pc}} % output
\newcommand{\upc}{\vo{u}_{pc}} % output
\newcommand{\Apc}{\mo{A}_{pc}} % A matrix
\newcommand{\Bpc}{\mo{B}_{pc}} % B matrix
\newcommand{\Cpc}{\mo{C}_{pc}} % C matrix

\newcommand{\I}[1]{\vo{I}_{{#1}}} % identity
\newcommand{\PHI}{\vo{\Phi}}
\newcommand{\phiKron}{(\PHI^T \otimes \I{n})}
%\renewcommand{\vec}[1]{\textbf{vec}\left({#1}\right)}
\renewcommand{\vec}[1]{\boldsymbol{\mathsf{vec}}\left({#1}\right)}

\newcommand{\X}{\mo{X}} % state
\newcommand{\F}{\mo{F}} 
\newcommand{\W}{\mo{W}} 
\newcommand{\K}{\mo{K}} % Feedback gain.

\newcommand{\set}[1]{\mathcal{#1}}
\newcommand{\inner}[1]{\left\langle #1 \right\rangle}
\newcommand{\Inner}[1]{\Bigl\langle #1 \Bigr\rangle}

% Labelling
% =======
\newcommand{\figlabel}[1]{\label{fig:#1}}
\newcommand{\eqnlabel}[1]{\label{eqn:#1}}
\newcommand{\seclabel}[1]{\label{sec:#1}}

% Referencing
% =========
%\newcommand{\eqn}[1]{eqn.(\ref{eqn:#1})}
\newcommand{\eqn}[1]{\eqref{eqn:#1}}
\newcommand{\Eqn}[1]{Eqn.(\ref{eqn:#1})}
\newcommand{\fig}[1]{Fig.(\ref{fig:#1})}
\newcommand{\Fig}[1]{Fig.(\ref{fig:#1})}
\newcommand{\Sec}[1]{\S \ref{sec:#1}}

% Misc
% ====
\newcommand{\etal}{\textit{et al. }}
\newcommand{\heading}[1]{\textcolor{darkgray}{\headingfont #1}}
%\newcommand{\heading}[1]{\textcolor{darkgray}{\textbf{#1}}}
%\newcommand{\heading}[1]{\textcolor{darkgray}{#1}}
%\setlength\parindent{0pt}

% Colors
% ======
\definecolor{darkgreen}{rgb}{0,0.65,0}
\newcommand{\blue}[1]{\textcolor{blue}{#1}}
\newcommand{\green}[1]{\textcolor{darkgreen}{#1}}
\newcommand{\comment}[1]{\textcolor{red}{#1}}
\newcommand{\sidenote}[1]{\textcolor{gray}{\text{\scriptsize #1}}}


% Redefinitions
% =============
%\renewcommand{\langle}{\left\langle} % For inner products.
%\renewcommand{\rangle}{\right\rangle} 

